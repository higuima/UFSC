\documentclass[]{article}
\usepackage[utf8]{inputenc}
\usepackage[portuguese]{babel}
\usepackage{graphicx}
\usepackage{indentfirst}
\begin{document}
\begin{titlepage}
    \begin{center}
        \vspace*{1cm}
        \huge
        \textbf{A equação do Milênio} % ou o projeto do milênio
        
        \large
        \vspace{0.5cm}
         A ponte que balançou a engenharia.
             
        \vspace{0.9cm}

        \includegraphics[width=0.8\textwidth]{logoUfsc.png}

        \vspace{0.8cm}
        \textbf{Helena Isola Guimarães}
        \vspace{0.8cm}
             
        Departmento de Matemática\\
        Universidade Federal de Santa Catarina\\
             
    \end{center}
\end{titlepage}

\begin{abstract}
    Neste trabalho, irei mostrar o problema de vibração forçada aplicado à uma ponte construída em Londres no ano 2000.
Este problema de vibração forçada pode ser descrito por uma equação diferencial linear não homogênea, que como vimos em aula,
pode ser resolvida pelo método dos coeficientes indeterminados.

Irei abordar também as implicações das constantes escolhidas para a EDO e como elas
geram resoluções diferentes para uma mesma equação. Além disso, farei um desenvolvimento 
focado no lado direito da EDO não homogênea e mostrar possíveis soluções tendo como base
o artigo (artigo que menciona equação do andar humano), que descreve o movimento oscilatório dos passos 
de uma multidão em uma equação trigonométrica.
\end{abstract}
\pagebreak 

\section{Introdução}

A ponte do milênio, foi construída em Londres no ano de 2000 para comemorar
a virada do milênio, com o objetivo de desafiar a engenharia e ter um estilo arquitetônico único, e foi exatamente isso que aconteceu.
Graças ao design moderno e desafiador da ponte, no dia da sua inauguração a ponte começou a se mostrar um projeto falho.
Enquanto as pessoas iam subindo na ponte e iam caminhando sobre ela, a ponte começou a balançar. Quanto mais pessoas andavam sobre ela,
mais ela balançava. Este incidente foi tão grave que tiveram que fechar a ponte até conseguirem consertar o problema.

A ponte sofria de um fenômeno denominado vibração lateral sincronizada (REFERENCIA AQUI), causado pelo andar das pessoas em cima da mesma.
Este imprevisto provocou dois anos de pesquisa sobre o caminhar humano e o impacto dele sobre a estrutura. 
Depois de uma extensa investigação, percebeu-se que os passos individuais não seriam capazes de gerar tanto impacto, e assim
o conceito de "efeito de multidão" se tornou foco do estudo também. Concluíram que, graças ao efeito de multidão,
as pessoas andando sobre a ponte sincronizavam o caminhar, mesmo sem perceber, e que, graças à movimentação horizontal extrema provocada pelos
passos da multidão, a ponte sofria o fenômeno de excitação lateral sincronizada.

Tendo em vista esse problema aplicado, este trabalho tem como objetivo descerver matemáticamente
o problema encontrado pelos engenheiros, mostrar que o problema é de fato, uma equação linear não 
homogênea de segunda ordem, resolver esta EDO e mostrar suas implicações no mundo real.
Na modelagem, irei discutir quais foram as causas deste problema e com essas variáveis, montar a EDO que
descreve o fenômeno. Na resolução, como indica o nome, irei utilizar do método dos coeficientes indeterminados
para resolver a EDO encontrada. Na discussão, irei demonstrar quais as consequências da resolução, suas representações
gráficas e o que significam.
Na conclusão irei discutir como esta ponte revolucionou a história da engenharia, da arquitetura, explicar quais foram as soluções 
aplicadas pelos engenheiros responsáveis e como este problema gerou uma nova maneira de pensar matemáticamente.

\pagebreak 
\section{Modelagem}

A vibração forçada que foi provocada pelo movimento horizontal lateral dos passos sincronizados das pessoas na ponte também pode ser 
chamado de ressônancia, pois a ressonância  é o resultado de quando a frequência aplicada reforça a frequência natural e nesse caso temos vibrações de grande amplitude.

Ao andar, o centro de gravidade das pessoas se alterna lateralmente. 
Quando temos apenas uma pessoa caminhando, este esforço lateral pode ser considerado 
desprezível, porém quando consideramos uma multidão de centenas de pessoas, temos o 
chamado “efeito de multidão”, em que a sincronia exercida pela caminhada do público é tal 
que acaba criando uma onda com frequência muito similar à da própria ponte. Assim, 
temos que a frequência natural da onda é incrementada e sua amplitude cresce, 
resultando assim em um fenômeno de ressonância que causou um movimento lateral 
sincronizado e altamente perceptível à todos em cima e à volta da ponte.

Pensando em vibrações de molas, podemos descrever a força elástica como $F_e = -k.y$, onde k é a
constante elástica e y é a unidade de distorção da mola. Pela segunda lei de newton, sabemos que a força de um
objeto é igual à sua massa vezes à sua aceleração de maneira que podemos descrever a aceleração de um objeto como:
\[ a = \frac{d^2 y}{d t^2} \]

Assim, temos que a equação de força elástica é:
\[ m.\frac{d^2 y}{d t^2} = -k.y \]
\[ m.\frac{d^2 y}{d t^2} + k.y = 0\]

Tendo em vista que a força de amortecimento é conhecida como $F_c = -c\frac{dy}{dt}$,
ao pensar em uma vibração amortecida temos que a força do objeto será a 
força restauradora + força de amortecimento
\[ m.\frac{d^2 y}{d t^2} = -k.y - c\frac{dy}{dt}\]
\[ m.\frac{d^2 y}{d t^2} + c\frac{dy}{dt}  + k.y = 0\]

Agora, supondo que além da força restauradora e da força de amortecimento, existe uma força externa sendo aplicada no
objeto, por exemplo uma multidão de passos sincronizados que gera uma movimentação lateral.
Com essas variáveis teríamos que:
\[ m.\frac{d^2 y}{d t^2} = -k.y - c\frac{dy}{dt} + F(t)\] 
\[ m.\frac{d^2 y}{d t^2} + c\frac{dy}{dt}  + k.y = F(t)\] 

Esta equação é a que descreve a vibração forçada e quando existe uma escassez de amortecimento,
descreve também o fenômeno da ressonância.
Pode se perceber que ela é uma equação linear não homogênea de segunda ordem e para encontrarmos 
uma solução para esse tipo de equação, precisamos utilizar o método dos coeficientes indeterminados.

\pagebreak 
\section{Resolução da EDO}
O método dos coeficientes indeterminados diz que a solução geral de uma equação 
diferencial não homogênea
da forma $y'' + p(t)'y + q(t)y = g(t)$ pode ser escrita como
\[ y(t) = Y(t) + c_1.y_1(t) + c_2.y_2(t)\] 
Para $c_1, c_2 \in R$ e onde \{$y_1,y_2$\} formam um conjunto fundamental de soluções
da EDO homogênea correspondente. Neste caso, temos que a EDO correspondente é uma equação
linear homogênea e é escrita como: 
\[y'' + p(t)y' + q(t)y = 0\] 

Aplicando esta ideia então à nossa equação linear não homogênea de segunda ordem que descreve o problema 
de vibração forçada da ponte, temos que:
A equação original será (adicionar label à equação)
\[  my'' + cy'  + ky = F(t)\] 
e sua equação correspondente será
\[ my'' + cy'  + ky = 0\]

Como a equação acima é linear e homogênea, iremos trabalhar com soluções exponenciais na forma $y=e^{rt}$, onde r é um parâmetro a ser determinado. (REFERENCIA BOYCE DIPRIMA)
Assim, temos que $y'=re^{rt}$  e $y''=r^2 e^{rt}$. Substituindo essas novas soluções na equação acima, podemos encontrar a chamada equação auxiliar,
que por $e^{rt}\neq 0$, poderemos assumir a equação (label) como equação auxiliar, ou característica:
\[ (mr^2 + cr  + k)e^{rt} = 0\] 
\[ mr^2 + cr  + k = 0\] 

Como temos uma equação de segundo grau como auxiliar, temos que as suas raízes serão:
\[ r_1 = \frac{-c + \sqrt{c^2 - 4mk}}{2m}\] \[ r_2 = \frac{-c - \sqrt{c^2 - 4mk}}{2m} \]
Portanto temos três possíveis cenários para avaliarmos:
\begin{enumerate}
    \item $c^2 - 4mk > 0$
    \item $c^2 - 4mk < 0$
    \item $c^2 - 4mk = 0$
\end{enumerate}

Em cada um desses casos temos consequências diretas para a vibração de um objeto e
diferentes resultados matemáticos como soluções da EDO. Iremos abordar
quais são esses resultados a seguir e suas implicações serão discutidas na próxima seção.
\pagebreak

- todos os deltas possíveis o que eles implicam

\subsection{Caso 1: discriminante maior do que zero}

Quando temos valores constantes de c, m e k tal que $c^2 - 4mk > 0$, podemos afirmar que 
$r_1$ e $r_2$ serão diferentes entre si. Sendo assim, a solução será uma combinação linear da forma:
\[y = c_1.e^{r_1t} + c_2.e^{r_2t}\]

Onde $c_1$ e $c_2$ são constantes que podem ser encontradas quando a EDO faz parte de um problema
de valor inicial, ou seja, quando temos valores iniciais de $y(0)=y_0$ e de $y'(0)=y_1$, como no 
exemplo 3 da seção 1 do capítulo 3 do livro Boyce DiPrima. 

É importante destacar que  $y_1 = e^{r_1t}$ e $y_2 = ^{r_2t}$.

\subsection{Caso 2: discriminante menor do que zero}

Quando temos valores constantes de c, m e k tal que $c^2 - 4mk < 0$, podemos afirmar que 
$r_1$ e $r_2$ serão raízes complexas e distintas entre si. Podemos descrever $r_1$ e $r_2$ como:
\[ r_1 = \alpha + i\beta\] 
\[ r_2 = \alpha - i\beta\] 
e portanto, a solução geral será:
\[ y(t) = c_1.e^{(\alpha + i\beta)t} + c_2.e^{(\alpha - i\beta)t}\]

Note que neste caso, a solução geral também é uma combinação linear de um conjunto fundamental de soluções,
onde $y_1 = e^{(\alpha + i\beta)t}$ e $y_2 = e^{(\alpha - i\beta)t}$.

(COMENTAR SOLUÇÕES trigonométricaS)

\subsection{Caso 3: discriminante igual a zero}

Quando temos valores constantes de c, m e k tal que $c^2 - 4mk = 0$, podemos afirmar que 
$r_1$ e $r_2$ serão correspondentes entre si. Este é o caso que descreve o nosso problema.



\subsection{O lado direito da equação}

- resolver a equação não homogênea para algum F(t): lado direito

\pagebreak

\section{Discussão dos Resultados}

- quais são as soluções da EDO para cada delta possível
- qual se aplica ao problema

\subsection{Superamortrcimento}

\subsection{Subamortecimneto}

\subsection{Amortecimento crítico}

\end{document}