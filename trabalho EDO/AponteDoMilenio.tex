\documentclass[]{article}
\usepackage[utf8]{inputenc}
\usepackage[portuguese]{babel}
\usepackage{graphicx}
\usepackage{indentfirst}
\begin{document}
\begin{titlepage}
    \begin{center}
        \vspace*{1cm}
        \huge
        \textbf{A equação do Milênio} % ou o projeto do milênio
        
        \large
        \vspace{0.5cm}
         A ponte que balançou a engenharia.
             
        \vspace{0.5cm}

        \includegraphics[width=1\textwidth]{logoUfsc.png}

        \vspace{0.8cm}
        \textbf{Helena Isola Guimarães}
        \vspace{0.8cm}
             
        Departmento de Matemática\\
        Universidade Federal de Santa Catarina\\
             
    \end{center}
\end{titlepage}

\begin{abstract}
    Neste trabalho, irei mostrar o problema de vibração forçada aplicado à uma ponte construída em Londres no ano 2000.
Este problema de vibração forçada pode ser descrito por uma equação diferencial linear não homogênea, que como vimos em aula,
pode ser resolvida pelo método dos coeficientes indeterminados.
\end{abstract}
\pagebreak 

\section{Introdução}

A ponte do milênio, construída em Londres no ano de 2000 para comemorar
a virada do milênio. Seu objetivo era desafiar a engenharia e ter um estilo arquitetônico único, e foi exatamente isso que aconteceu.
Graças ao design moderno e desafiador da ponte, no dia da sua inauguração a ponte começou a se mostrar um projeto falho.
Enquanto as pessoas iam subindo na ponte e iam caminhando sobre ela, a ponte começou a balançar. Quanto mais pessoas andavam sobre ela,
mais ela balançava. Este incidente foi tão grave que tiveram que fechar a ponte até conseguirem consertar o problema.

A ponte sofria de um fenômeno denominado vibração lateral sincronizada, causado pelo andar das pessoas em cima da mesma.
Este imprevisto provocou dois anos de pesquisa sobre o caminhar humano e o impacto dele sobre a estrutura. 
Depois de uma extensa investigação, percebeu-se que os passos individuais não seriam capazes de gerar tanto impacto, e assim
o conceito de "efeito de multidão" se tornou foco do estudo também. Concluíram que, graças ao efeito de multidão,
as pessoas andando sobre a ponte sincronizavam o caminhar, mesmo sem perceber, e que, graças à movimentação horizontal extrema provocada pelos
passos da multidão, a ponte sofria o fenômeno de excitação lateral sincronizada.

Tendo em vista esse problema aplicado, este trabalho tem como objetivo descerver matemáticamente
o problema encontrado pelos engenheiros, mostrar que o problema é de fato, uma equação linear não 
homogênea de segunda ordem, resolver esta EDO e mostrar suas reais implicações no mundo real.
Na modelagem, irei discutir quais foram as causas deste problema e com essas variáveis, montar a EDO que
descreve o fenômeno. Na resolução, como indica o nome, irei utilizar do método dos coeficientes indeterminados
para resolver a EDO encontrada. Na discussão, irei demonstrar quais as consequências da resolução, suas representações
gráficas e o que significam e explicar quais foram as soluções aplicadas pelos engenheiros responsáveis.
Na conclusão irei discutir como esta ponte revolucionou a história da engenharia, da arquitetura e 
como este problema gerou uma nova maneira de pensar matemáticamente.

\pagebreak 
\section{Modelagem}

A vibração forçada que foi provocada pelo movimento horizontal lateral dos passos sincronizados das pessoas na ponte também pode ser 
chamado de ressônancia, pois a ressonância  é o resultado de quando a frequência aplicada reforça a frequência natural e nesse caso temos vibrações de grande amplitude.

Ao andar, o centro de gravidade das pessoas alternava lateralmente. 
Quando temos apenas uma pessoa caminhando, este esforço lateral pode ser considerado 
desprezível, porém quando consideramos uma multidão de centenas de pessoas, temos o 
chamado “efeito multidão”, em que a sincronia exercida pela caminhada do público é tal 
que acaba criando uma onda com frequência muito similar à da própria ponte. Assim, 
temos que a frequência natural da onda é incrementada e sua amplitude cresce, 
resultando assim em um fenômeno de ressonância que causou um movimento lateral 
sincronizado e altamente perceptível à todos em cima e à volta da ponte.

Pensando em vibrações de molas, podemos descrever a força elástica como $F_e = -k.y$, onde k é a
constante elástica e x é a unidade de distorção da mola. Pela segunda lei de newton, sabemos que a força de um
objeto é igual à sua masssa vezes à sua aceleração de maneira que podemos descrever a aceleração de um objeto como:
\[ a = \frac{d^2 y}{d t^2} \]

Assim, temos que a equação de força elástica é:
\[ m.\frac{d^2 y}{d t^2} = -k.y \]
\[ m.\frac{d^2 y}{d t^2} + k.y = 0\]

Tendo em vista que a força de amortecimento é conhecida como $F_c = -c\frac{dy}{dt}$,
ao pensar em uma vibração amortecida temos que a força do objeto será a 
força restauradora + força de amortecimento
\[ m.\frac{d^2 y}{d t^2} = -k.y - c\frac{dy}{dt}\]
\[ m.\frac{d^2 y}{d t^2} + c\frac{dy}{dt}  + k.y = 0\]

Agora, supondo que além da força restauradora e da força de amortecimento, existe uma força externa sendo aplciada no
objeto, por exemplo uma multidão de passos sincronizados que gera uma movimentação lateral.
Com essas variáveis teríamos que:
\[ m.\frac{d^2 y}{d t^2} = -k.y - c\frac{dy}{dt} + F(t)\] 
\[ m.\frac{d^2 y}{d t^2} + c\frac{dy}{dt}  + k.y = F(t)\] 

Esta equação é a que descreve a vibração forçada e também, quando existe um subamortecimento (ou c=0),
temos também a descrição do fenômeno da ressonância.
Pode se perceber que ela é uma equação linear não homogênea de segunda ordem e para encontrarmos 
uma solução para esse tipo de equação, precisamos utilizar o método dos coeficientes indeterminados.

\pagebreak 
\section{Resolução da EDO}
O método dos coeficientes indeterminados diz que a solução geral de uma equação diferencial não homogênea
da forma $y'' + p(t)'y + q(t)y = g(t)$ pode ser escrita como
\[ y(x) = Y(t) + c_1.y_1(t) + c_2.y_2(t)\] 
Para $c_1, c_2 \in R$ e onde ${y_1,y_2}$ formam um conjunto fundamental de soluções
da EDO homogênea correspondente à equação. Neste caso, temos que a EDO correspondente é uma equação
linear homogênea e é escrita como: 
\[y'' + p(t)'y + q(t)y = 0\] 

Aplicando esta ideia então à nossa equação linear não homogênea de segunda ordem que descreve o problema 
de vibração forçada da ponte, temos que:
A equação original será 

\[  my'' + cy'  + ky = F(t)\] 
e sua equação correspondente será
\[ my'' + cy'  + ky = 0\]
Agora, podemos encontrar uma equação auxiliar e suas raízes:
\[ mr^2 + cr  + k = 0\] 

\vspace*{0.8cm}
descrever:

- todos os deltas possíveis o que eles implicam

- quais são as soluções da EDO para cada delta possível

- qual se aplica ao problema

- resolver a equação não homogênea para algum F(t)
\end{document}