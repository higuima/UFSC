\documentclass[]{article}
\usepackage[utf8]{inputenc}
\usepackage[portuguese]{babel}
\usepackage{graphicx}
\begin{document}
\begin{titlepage}
    \begin{center}
        \vspace*{1cm}
        \huge
        \textbf{A equação do Milênio} % ou o projeto do milênio
        
        \large
        \vspace{0.5cm}
         A ponte que balançou a engenharia.
             
        \vspace{0.5cm}

        \includegraphics[width=1\textwidth]{logoUfsc.png}

        \vspace{0.8cm}
        \textbf{Helena Isola Guimarães}
        \vspace{0.8cm}
             
        Departmento de Matemática\\
        University Federal de Santa Catarina\\
             
    \end{center}
\end{titlepage}

\section{Introdução}
Neste trabalho, iremos abordar a ponte do milênio, construída em Londres no ano de 2000 para comemorar
a virada do milênio. Seu objetivo era desafiar a engenharia e ter um estilo arquitetônico único, e foi exatamente isso que aconteceu.
Graças ao design moderno e limitado da ponte, no dia da sua inauguração a ponte começou a se mostrar um projeto falho.
Enquanto as pessoas iam subindo na ponte e iam caminhando sobre ela, a ponte começou a balançar. Quanto mais pessoas andavam sobre ela,
mais ela balançava. Este incidente foi tão grave que tiveram que fechar a ponte até conseguirem consertar o problema.

Consertar o problema acabou se tornando um grande desafio aos engenheiros que haviam construído ela em primeiro lugar, pois acabaram demorando
dois anos até encontrarem uma solução que além de impedir a ponte de balançar, ainda estivesse no mesmo contexto moderno no qual a arquitetura dela almejava.


\end{document}