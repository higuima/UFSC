\documentclass{article}
\usepackage[utf8]{inputenc}
\usepackage[portuguese]{babel}
\begin{document}
Este é um pequeno documento para ilustrar o uso básico de LaTeX.

Simplesmente deixe uma linha em branco para gerar um novo parágrafo; identação é
automatica.

Expressões matemáticas como  $y = 3\sin x$ são obtidas com o símbolo do cifrão.
Equações podem ser expostas como em 
\[ y = 3\sin x.\]
Também é possível enumerar equações:
\begin{equation}\label{equal}
y’ =3\cos x.
\end{equation}
Como nós identificamos a equação acima com um enumerador, podemos nos referir
à ela sem ter que saber qual é o número que foi utilizado. Assim, a equação anterior foi enumerada como~(\ref{equal}).
Potências (superescritas), como em $x^2$, são obtidas com \verb"^";
potências mais complicadas precisam ser colocadas dentro de chaves: $x^{2+\alpha}$.
Assim como, subscritos são obtidos com o underline: $y_3$ ou $y_{n+1}$.
Usamos ambas maneiras em $x_{n+1}^{2+\alpha}$.
\end{document}